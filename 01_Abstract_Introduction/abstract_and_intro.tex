
\section*{Abstract}

The Spaciotemporal Vortex Model (SVM) offers a physically grounded framework in which time is reinterpreted as a layered fourth spatial dimension. Temporal evolution results from energy transitions between discrete temporal layers. Unlike conventional cosmological models, which treat time as a continuous linear axis and fail to address the low-entropy initial state, arrow of time, or dark phenomena, SVM introduces a novel mechanism: gravitational resistance to inter-layer transitions. This reinterpretation provides a thermodynamically coherent basis for cyclic cosmological dynamics, entropy management, and information encoding via curvature. The model derives formal evolution laws, thermodynamic couplings, and quantization paths, offering multiple empirical implications for entropy collapse, dark energy behavior, and curvature-entropy correspondence.


\section{Introduction}

Contemporary cosmological models based on General Relativity conceptualize time as a linear, continuous dimension, fundamentally distinct from space. However, they leave unresolved several foundational problems:

\begin{itemize}
    \item The origin of the arrow of time,
    \item The mechanism behind the universe's extremely low initial entropy,
    \item The nature of dark matter and dark energy,
    \item The inability to naturally implement cyclic or entropy-resetting cosmologies.
\end{itemize}

The Spaciotemporal Vortex Model (SVM) introduces a restructured interpretation of time as a layered, discretized spatial dimension. Energy transitions between these temporal layers define temporal progression. Gravitational interactions are reinterpreted as impedance to such transitions. This framework offers a thermodynamically coupled, cyclic cosmology that is ontologically novel, mathematically formalized, and empirically tractable.
