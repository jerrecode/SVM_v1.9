
\section{Conclusion and Outlook}

The Spaciotemporal Vortex Model (SVM) provides a physically motivated and mathematically consistent alternative to conventional cosmological models by redefining time as a layered spatial dimension. This ontological shift allows energy and entropy to evolve along discrete temporal transitions, which are interpreted as physical processes rather than abstract parameters.

\subsection*{Empirical Priorities}

Key avenues for future empirical investigation include:
\begin{itemize}
    \item Detection of entropy-curvature correlations in the CMB.
    \item Searching for discrete imprints or "layer scars" in gravitational wave spectra.
    \item Investigating temperature fluctuations tied to quantum entropy effects.
\end{itemize}

\subsection*{Computational Tasks}

We aim to simulate:
\begin{itemize}
    \item Layered state evolutions using finite-difference or finite-element methods.
    \item Entropy dispersion as geometric diffusion in $\Omega_n$.
    \item Free energy landscapes and curvature-induced state changes.
\end{itemize}

\subsection*{Theoretical Development}

Future refinements of the model will involve:
\begin{itemize}
    \item Tensorial formulations of $g_{ab}$ with curvature invariants.
    \item Deriving gravitational field equations based on thermodynamic conjugates.
    \item Incorporating black hole boundary layers as entropy flux gates.
\end{itemize}

SVM proposes a coherent language that bridges thermodynamic irreversibility, quantum uncertainty, and geometric structure. Continued development will focus on rendering this framework testable, implementable, and compatible with fundamental physical constraints.
